Fluorescence detectors are often used to examine the energy spectrum and arrival directions of ultra high energy cosmic rays. An accurate reconstruction of geometry is key when determining both energy and direction. In the past, it has been necessary to build multiple fluorescence detectors to take advantage of the improved reconstruction of a stereo observation. In this work, we investigate a strategy which uses Cherenkov radiation to improve accuracy without the cost of multiple detectors. Cherenkov radiation follows the path of the cosmic ray cascade in a tight cone, producing a bright point of light where the shower hits the ground. The ground impact is a fixed point along the shower track and can be used in fitting. We investigate the Cherenkov-assisted approach using Monte Carlo simulations of cascade light production, detection, and reconstruction. We also simulate individual case studies to gain a better understanding of typical profiles. We find that for a detector \SI{200}{m} above the ground with resolution \ang{0.086}, the Cherenkov reconstruction gives an improvement in about 45 percent of cases. In these cases, the mean absolute error in the impact parameter is reduced from 15.6 percent to 10.7 percent.