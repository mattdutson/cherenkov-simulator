The simulation makes a number of simplifying assumptions, one of which is a lack of atmospheric scattering. Stratton describes the two main mechanisms, Rayleigh and aerosol scattering \cite{stratton2012ta}. Both give some attenuation between the shower and the detector. Assuming this attenuation affects fluorescence and scattered Cherenkov equally, while the number of detected photons would tend to be lower, the relative heights of the fluorescence and Cherenkov components would be mostly unchanged. The lack of simulated attenuation leads to an overestimate of the limits of the aperture, although the extent of the overestimate is unclear. More important is the scattering of the Cherenkov beam as it travels to the ground. Such scattering is significant enough that other Telescope Array simulations account for it. The effect is twofold. The intensity of the Cherenkov beam is reduced, while the apparent intensity of the fluorescence track is increased by Cherenkov radiation scattered directly into the detector.

The simulation assumes that all light striking the ground is reflected. While this may be realistic for a surface covering like snow, the reflection coefficient would likely be substantially lower for soil or vegetation. The assumption of ideal, diffuse reflection is also made. In some real-world mix of diffuse and specular reflectance, the reflected intensity would tend to be higher both near the surface normal and the direction of a specularly reflected ray, leading to a general reduction in observed Cherenkov reflection. Due both to these reflectance approximations and to the lack of simulated scattering, the ground point dimness problem discussed in Section \ref{sec:mc_results} would be increased in a real detector. This indicates that we may have overestimated the size of the Cherenkov parameter space.

Light production depends heavily on local atmospheric properties including pressure and density. Our exponential atmosphere is an approximation of the more physically accurate US Standard Atmosphere. The impact of this approximation varies with altitude, and it is unclear whether it skews light production in a particular direction. For instance, our constant temperature estimate of $T = \SI{273}{K}$ is likely high. Equation \ref{eq:fluorescence} shows that this overestimate of temperature leads to an underestimate of fluorescence production.

The performance of the Cherenkov technique could likely be improved by rethinking the ground impact algorithm. Recall that currently, this algorithm performs a simple search over the ground to find the brightest pixel. This makes poor use of the available information. For example, if the impact point is on the boundary of two pixels, simply choosing the brighter of the two could cause the impact direction to be off by as much as half a resolution increment. A better approach could be to take a weighted average over pixels immediately adjacent to the brightest. In addition, information about the arrival time of the Cherenkov flash could be used to constrain the distance of the impact point relative to the rest of the shower.

Much of the cost of a hybrid Cherenkov detector would be in the individual silicon photomultipliers. Assuming a per-pixel cost of \$50 and the simulated array of \num{71284} pixels, the total cost would be on the order of \$3.5 million. However, this could be reduced by only using high-resolution pixels below the horizon. Only \num{26696} pixels of the simulated array are below the horizon, allowing for a potential cost reduction of over sixty percent. This would, however, lead to additional complexity in the design of the electronics.